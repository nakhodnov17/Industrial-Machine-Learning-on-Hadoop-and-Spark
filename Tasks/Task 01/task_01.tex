\documentclass[10pt,fleqn]{article}

\usepackage[english]{babel}
\usepackage[utf8]{inputenc}
\usepackage{color}
\usepackage{amsmath}
\usepackage{amssymb}
\usepackage{graphics}
\usepackage{epsfig}
\usepackage{bm}
\usepackage[colorlinks,urlcolor=blue]{hyperref}
\usepackage{tikz}
\usepackage{pgfplots}
\usepackage{verbatim}
\usepackage{mdframed}
\usepackage{dirtree}
\usepackage{indentfirst}
\usepackage{url}
\usepackage{float}

\definecolor{codegray}{gray}{0.9}
\newcommand{\code}[1]{%
  \begingroup\setlength{\fboxsep}{1pt}%
  \colorbox{codegray}{\texttt{\hspace*{2pt}\vphantom{Ay}#1\hspace*{2pt}}}%
  \endgroup
}

% mdinlinecode command for including code snippets inline
% (fake verbatim, so all special character should be escaped,
% or textmode equivalents of special characters should be used)
\definecolor{mdinlinecodeboxframecolor}{HTML}{DDDDDD}
\definecolor{mdinlinecodeboxbackgroundcolor}{HTML}{F8F8F8}
\newcommand{\mdinlinecode}[1]{%
    \begin{tikzpicture}[baseline=0ex]%
        \node[anchor=base,%
            text height=0.9em,%
            text depth=0.9ex,%
            inner ysep=0pt,%
            draw=mdinlinecodeboxframecolor,%
            fill=mdinlinecodeboxbackgroundcolor,%
            rounded corners=1.5pt] at (0,0) {\small\texttt{#1}};%
    \end{tikzpicture}%
}

\newmdenv[font=\footnotesize,%
linewidth=0.4pt,%
roundcorner=2pt,%
linecolor=mdinlinecodeboxframecolor,%
backgroundcolor=mdinlinecodeboxbackgroundcolor,%
settings={\setlength{\parindent}{0pt}}]{mdcdblk}
\newenvironment{mdcodeblock}{\endgraf\verbatim}{\endverbatim}
\BeforeBeginEnvironment{mdcodeblock}{\begin{mdcdblk}}
\AfterEndEnvironment{mdcodeblock}{\end{mdcdblk}}

\textheight=26cm % text height
\textwidth=18cm % text width
\oddsidemargin=-1cm % left margin
\topmargin=-3cm % top margin
\sloppy

\newcounter{example}

%-- Vectors in bold
\def\vec#1{\mathchoice{\mbox{\boldmath$\displaystyle#1$}}
{\mbox{\boldmath$\textstyle#1$}} {\mbox{\boldmath$\scriptstyle#1$}} {\mbox{\boldmath$\scriptscriptstyle#1$}}}

\DeclareMathOperator{\B}{Bin}
\DeclareMathOperator{\Ps}{Poiss}
\DeclareMathOperator{\R}{Unif}
\DeclareMathOperator{\sign}{\mathrm{sign}}
\DeclareMathOperator{\softmax}{\mathrm{softmax}}
\DeclareMathOperator{\loss}{\mathcal{L}}

\pagestyle{empty}

\begin{document}

\begin{center}
    \begin{tabular}{|p{17.5cm}|}
        \hline
        \begin{center} \Large Task 01. Hadoop. HDFS. \end{center}\\
        \textbf{Industrial Machine Learning on Hadoop and Spark, Fall 2025}\\
        \hline
    \end{tabular}
\end{center}

\begin{tabbing}
Task start date: September 21, 2025, 23:59CET.\\
Hard Deadline: \textcolor{red}{\bf September 28, 2025, 23:59CET.}
\end{tabbing}

\section*{Assignment Statement}
This assignment is aimed at introducing you to the Apache Hadoop infrastructure.

The task consists of the single part:
\begin{enumerate}
\item Performing basic actions in the HDFS filesystem
\end{enumerate}

\begin{section}{Basic actions in HDFS (5\% for each point + 15\% for questions)}
In this section, you need to write a script \mdinlinecode{hdfs.sh} that performs the following sequence of actions from the root directory of the Hadoop cluster’s \mdinlinecode{namenode}:
\begin{enumerate}
\item Create a local file \mdinlinecode{test.txt} with a size of $100$Mb
\item Create HDFS directories \mdinlinecode{temp} and \mdinlinecode{logs}
\item Upload the file \mdinlinecode{test.txt} into the \mdinlinecode{temp} directory
\item View the properties of the uploaded file
\item Move the file \mdinlinecode{test.txt} into the \mdinlinecode{logs} directory
\item Set the replication factor for the file to $1$
\item Copy \mdinlinecode{test.txt} to \mdinlinecode{test2.txt}
\item Copy the directory \mdinlinecode{logs} into \mdinlinecode{logs2} using \mdinlinecode{hadoop distcp}
\item Set file permissions to read and write only for the owner for \mdinlinecode{test2.txt} in the \mdinlinecode{logs2} directory
\item Display the properties of all files in \mdinlinecode{logs2}
\item View the size of all directories in \mdinlinecode{/}
\item Delete the \mdinlinecode{logs} directory
\item Run \mdinlinecode{fsck} on the \mdinlinecode{logs2} directory
\item View the HDFS report via \mdinlinecode{dfsadmin}
\item Move \mdinlinecode{/logs2/test2.txt} to the local folder \mdinlinecode{/}
\item Append the contents of the local file \mdinlinecode{test2.txt} to the end of the file \mdinlinecode{/logs2/test.txt} in HDFS
\item Output the size of each file in \mdinlinecode{/logs2} in Mb
\end{enumerate}

Each step must correspond to a single command. The execution result (stdout, stderr) of \mdinlinecode{hdfs.sh} must be saved in the file \mdinlinecode{hdfs.output.txt}.

Also, answer the following questions in the file \mdinlinecode{hdfs.answers.md}:
\begin{enumerate}
\item How many block replicas are missing after running the \mdinlinecode{fsck} command? Explain the reason for the missing replicas. \textbf{(7\%)}
\item What is the size of the HDFS filesystem? \textbf{(8\%)}
\end{enumerate}

As your submission, you must provide three files named exactly as follows: \mdinlinecode{hdfs.sh}, \mdinlinecode{hdfs.output.txt}, \mdinlinecode{hdfs.answers.md}.
\end{section}

\end{document}
